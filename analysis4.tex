\documentclass{notes}

\author{\sffamily{\Large{Михаил Пирогов}} \\ \sffamily{\Large{записал со слов лектора А. А. Лодкина}}}
\title{\sffamily{\Huge{Анализ, 4 семестр}}}

\begin{document}
	\maketitle
	\chapter{Теория меры}

	\section{Алгебры и \texorpdfstring{$\sigma$}{σ}-алгебры множеств.}

	\begin{de}
		Пусть $X$~--- некоторое множество. Тогда $\mc{A} \subset 2^{X}$ называется \ti{алгеброй,} если выполняются следующие условия:
		\begin{enumerate}
			\item $\varnothing, \, X \in \mc{A},$
			\item $A, \, B \in \mc{A} \so A \cup B, \, A \cap B, \, A \setminus B \in \mc{A}.$
		\end{enumerate} 
	\end{de}

	\begin{exc}
		Пусть $\mc{A} \subset 2^X$~--- алгебра, $|\mc{A}| < \infty$. Тогда $|\mc{A}| = 2^n$ для некоторого $n$.
		\begin{proof}
			Так как $X \in \mc{A},$ каждый элемент $X$ содержится как минимум в одном элементе $\mc{A}$. Пусть $A(x)$~--- пересечение всех множеств из $\mc{A},$ содержащих $x$. Понятно, что $A(x)$ непусто, т.к. $x \in A(x)$. Разобьём дальнейшее доказательство на несколько пунктов:
			\begin{enumerate}
				\item Мы определили $A(x),$ как наименьшее по включению множество, удовлетворяющее некоторому свойству. Поэтому у него есть эквивалентное определение: $A(x)$~--- такое множество, что если $x \in B \in \mc{A},$ то $A(x) \subset B$ \footnote{Заметим, что мы существенно испльзуем конечность $\mc{A}$ каждый раз, когда говорим, что $A(x) \in \mc{A}$!}.
				\item Введём отношение на $X\col$ пусть $x \sim y,$ если $A(x) = A(y)$. Очевидно, что это отношение эквивалентности. Докажем, что $x \sim y \eqv y \in A(x)$.

				Пусть $y \in A(x)$. Предположим, что $A(y) \neq A(x)$. Тогда выполняется минимум одно из двух утверждений: либо $A(y)$ содержит элемент, которого нет в $A(x),$ либо наоборот. Пусть первое. Тогда $B = A(x) \cap A(y)$~--- элемент $\mc{A},$ который содержит $y$ и строго меньше $A(y),$ чего не может быть. Пусть второе. Тогда если $A(y)$ не содержит $x,$ то $A(x) \setminus A(y)$ является элементом $\mc{A},$ содержащим $x,$ а если содержит, то снова $A(x) \cap A(y)$ является таким элементом. Причём строго меньшим, чем $A(x),$ что опять ведёт нас к противоречию.

				Пусть $A(x) = A(y)$. Предположим, что $y \notin A(x)$. Но тогда $y \notin A(y),$ что точно ложь.
				\item Разобьём $X$ на классы эквивалентности по отношению $\sim\scol$ обозначим множество этих классов $\hat{\mc{A}}$. Понятно, что $|\hat{\mc{A}}| < \infty,$ ведь $\hat{\mc{A}} \subset \mc{A}$. Пусть $B \in \mc{A}$ и $\hat{B} \in \hat{\mc{B}}$. Докажем, что если $B \cap \hat{B} \neq \varnothing,$ то $B \cap \hat{B} = \hat{B}$.

				Предположим противное: пусть $x \in B \cap \hat{B}$ и $y \in \hat{B} \setminus B$. Из определения отношения эквивалентности понятно, что $\hat{B} = A(x) = A(y)$. Но заметим тогда, что $\hat{B} \setminus B$~--- множество из $\mc{A},$ содержащее $y$ и строго меньшее $\hat{B},$ чего не может быть.
				\item Из сделанного нетрудно увидеть, что любое $B \in \mc{A}$ можно представить, как объединение множеств из $\hat{\mc{A}}\col$ просто для каждого $b \in B$ взять $A(b)$ и объединить их все. При этом понятно, что любое объединение множеств из $\hat{\mc{A}}$ лежит в $A$. Т.к. элементы $\hat{\mc{A}}$ не пересекаются, нетрудно увидеть, что отображение, сопоставляющее множеству $\mc{B} \subset \hat{\mc{A}}$ объединение всех его элементов есть биекция~--- биекция между множествами $2^{\hat{\mc{A}}} и \mc{A}$. Поэтому количество элементов $\mc{A}$ имеет искомый вид.
			\end{enumerate}
		\end{proof}
	\end{exc}
\end{document}

