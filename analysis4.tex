\documentclass{notes}

\author{\sffamily{\Large{Михаил Пирогов}} \\ \sffamily{\Large{записал со слов лектора А. А. Лодкина}}}
\title{\sffamily{\Huge{Анализ, 4 семестр}}}

\DeclareMathOperator{\Cell}{Cell}

\begin{document}
	\maketitle
	\chapter{Теория меры}

	\section{Алгебры и \texorpdfstring{$\sigma$}{σ}-алгебры множеств.}

	\begin{de}
		Пусть $X$~--- некоторое множество. Тогда $\mc{A} \subset 2^{X}$ называется \ti{алгеброй,} если выполняются следующие условия:
		\begin{enumerate}
			\item $\varnothing, \, X \in \mc{A},$
			\item $A, \, B \in \mc{A} \so A \cup B, \, A \cap B, \, A \setminus B \in \mc{A}.$
		\end{enumerate} 
	\end{de}

	\begin{exc}
		Пусть $\mc{A} \subset 2^X$~--- алгебра, $|\mc{A}| < \infty$. Тогда $|\mc{A}| = 2^n$ для некоторого $n$.
		\begin{proof}
			Так как $X \in \mc{A},$ каждый элемент $X$ содержится как минимум в одном элементе $\mc{A}$. Пусть $A(x)$~--- пересечение всех множеств из $\mc{A},$ содержащих $x$. Понятно, что $A(x)$ непусто, т.к. $x \in A(x)$. Разобьём дальнейшее доказательство на несколько пунктов:
			\begin{enumerate}
				\item Мы определили $A(x),$ как наименьшее по включению множество, удовлетворяющее некоторому свойству. Поэтому у него есть эквивалентное определение: $A(x)$~--- такое множество, что если $x \in B \in \mc{A},$ то $A(x) \subset B$ \footnote{Заметим, что мы существенно испльзуем конечность $\mc{A}$ каждый раз, когда говорим, что $A(x) \in \mc{A}$!}.
				\item Введём отношение на $X\col$ пусть $x \sim y,$ если $A(x) = A(y)$. Очевидно, что это отношение эквивалентности. Докажем, что $x \sim y \eqv y \in A(x)$.

				Пусть $y \in A(x)$. Предположим, что $A(y) \neq A(x)$. Тогда выполняется минимум одно из двух утверждений: либо $A(y)$ содержит элемент, которого нет в $A(x),$ либо наоборот. Пусть первое. Тогда $B = A(x) \cap A(y)$~--- элемент $\mc{A},$ который содержит $y$ и строго меньше $A(y),$ чего не может быть. Пусть второе. Тогда если $A(y)$ не содержит $x,$ то $A(x) \setminus A(y)$ является элементом $\mc{A},$ содержащим $x,$ а если содержит, то снова $A(x) \cap A(y)$ является таким элементом. Причём строго меньшим, чем $A(x),$ что опять ведёт нас к противоречию.

				Пусть $A(x) = A(y)$. Предположим, что $y \notin A(x)$. Но тогда $y \notin A(y),$ что точно ложь.
				\item Разобьём $X$ на классы эквивалентности по отношению $\sim\scol$ обозначим множество этих классов $\hat{\mc{A}}$. Понятно, что $|\hat{\mc{A}}| < \infty,$ ведь $\hat{\mc{A}} \subset \mc{A}$. Пусть $B \in \mc{A}$ и $\hat{B} \in \hat{\mc{B}}$. Докажем, что если $B \cap \hat{B} \neq \varnothing,$ то $B \cap \hat{B} = \hat{B}$.

				Предположим противное: пусть $x \in B \cap \hat{B}$ и $y \in \hat{B} \setminus B$. Из определения отношения эквивалентности понятно, что $\hat{B} = A(x) = A(y)$. Но заметим тогда, что $\hat{B} \setminus B$~--- множество из $\mc{A},$ содержащее $y$ и строго меньшее $\hat{B},$ чего не может быть.
				\item Из сделанного нетрудно увидеть, что любое $B \in \mc{A}$ можно представить, как объединение множеств из $\hat{\mc{A}}\col$ просто для каждого $b \in B$ взять $A(b)$ и объединить их все. При этом понятно, что любое объединение множеств из $\hat{\mc{A}}$ лежит в $A$. Т.к. элементы $\hat{\mc{A}}$ не пересекаются, нетрудно увидеть, что отображение, сопоставляющее множеству $\mc{B} \subset \hat{\mc{A}}$ объединение всех его элементов есть биекция~--- биекция между множествами $2^{\hat{\mc{A}}}$ и $\mc{A}$. Поэтому количество элементов $\mc{A}$ имеет искомый вид.
			\end{enumerate}
		\end{proof}
	\end{exc}

	Примеры привести не очень сложно, не будем здесь на этом останавливаться.

	\begin{de}
		\ti{$\sigma$-алгеброй} называется алгебра, замкнутая относительно счётных объединений и пересечений.
	\end{de}

	\begin{de}
		Пусть $\mc{E} \subset 2^X.$ Тогда наименьншая $\sigma$-алгебра, содержащая $\mc{E},$ называется \ti{борелевской оболочкой} $\mc{E}$ и обозначается $\sigma(\mc{E})$. (Ссылаясь на факт, который уже упоминался в упражнении, заметим, что $\sigma(\mc{E})$ совпадает с пересечением всех $\sigma$-алгебр, содержащих $E$).
	\end{de}

	\begin{lm}
		Если $\mc{E}_2 \subset \sigma(\mc{E}_1),$ то $\sigma(\mc{E}_2) \subset \sigma(\mc{E}_1)$.
		\begin{proof}
			Из определения борелевской оболочки понятно, что 
			\[
				\mc{E}_2 \subset \sigma(\mc{E}_1) \so \sigma(\mc{E}_2) \subset \sigma(\sigma(\mc{E}_1)).
			\]
			При этом понятно, что правая часть равна $\sigma(\mc{E}_1),$ чего нам и надо.
		\end{proof}
	\end{lm}

	\section{Борелевская \texorpdfstring{$\sigma$}{σ}-алгебра.}

	\begin{de}
		Пусть $\mc{O}_n$~--- множество всех открытых множеств в $\R^n$. Тогда $\sigma$-алгебра $\sigma(\mc{O}_n)$ называется \ti{борелевской}.
	\end{de}

	\begin{de}
		Назовём \ti{$n$-мерной ячейкой} такое подмножество $\R^n\col$
		\begin{align*}
			n &= 1 \so \Delta = \begin{cases}
				[a, \, b), \; [a, \, \infty)\scol \\
				(-\infty, \, b), \; (-\infty, \, \infty)\scol 
			\end{cases} \\
			n &> 1 \so \Delta = \prod_{i = 1}^k \Delta_i,
		\end{align*}
		где $\Delta_i$~--- одномерные ячейки.
	\end{de}

	\begin{de}
		Назовём \ti{$n$-мерной алгеброй ячеек} множество
		\[
			\Cell_n = \left\{\bigcup_{i = 1}^k \Delta_i \, \bigg| \, k \in \N \right\},
		\]
		где $\Delta_i$~--- ячейки.
	\end{de}

	\begin{st}
		$\Cell_n$~--- действительно алгебра.
		\begin{proof}
			Чтобы сделать, нужно увидеть, что пересечение ячеек~--- ячейка, а потом представить пересечение объединений, как объединение пересечений.
		\end{proof}
	\end{st}

	\begin{thm}
		$\sigma(\Cell_n) = \sigma(\mc{O}_n)$.
		\begin{proof}
			Зная последний результат из предыдущего билета, имеем возможность доказывать, что
			\[
				\Cell_n \subset \sigma(\mc{O}_n) \text{ и } \mc{O}_n \subset
				 \sigma(\Cell_n).
			\]
			Это даст нам утверждение теоремы.

			Первое включение очевидно: можно представить любую ячейку, как пересечение вложенных прямоугольников, например. Поэтому и с объединением проблем не будет.

			Чтобы доказать второе, рассмотрим сначала ячейки с целыми вершинами, назовём их ячейками первого ранга. Побив каждую из них на $2^n$ частей (поделив каждую сторону на 2), получим ячейки второго ранга, продолжая процесс~--- ячейки ранга $n$. Пусть $U$~--- произвольное открытое множество, а $U_k$--- объединение всех ячеек ранга $k,$ пересекающих $U$.

			Рассмотрим $x$~--- произвольную точку не из $U$. Т.к. $U$ открыто, существует такое $\varepsilon,$ что  \[
				B_{\varepsilon}(x) \cap U = \varnothing.
			\]
			Заметим однако, что если ячейка ранга $k,$ то её сторона равна $2^{1 - k},$ а значит, диагональ~---
			\[
				\sqrt{n} 2^{1 - k}.
			\]
			Эта последовательность стремится к нулю при $k$ стремящемся к бесконечности, поэтому можно сделать так, что диагональ ячейки станет меньше, чем $\varepsilon,$ при всех $k > K$. Из этого будет следовать, что при $k > K \; x \notin U_k$.

			Отсюда следует, что
			\[
				U = \bigcap_{k = 1}^{\infty} U_k \so U \in \sigma(\Cell_n) \so \mc{O} \subset \sigma(\Cell_n). 
			\]
		\end{proof}
	\end{thm}

	\begin{st}
		Борелевской $\sigma$-алгебре принадлежат множества следующих типов:
		\begin{enumerate}
			\item Точки.
			\item Открытые, замкнутые.
			\item Не более чем счётные.
			\item Счётные пересечения открытых множеств~--- множества типа $G_{\delta}$.
			\item Счётные объединения замкнутых~--- множества типа $F_{\sigma}$.
			\item Счётные объединения множеств типа $G_{\delta}$~--- множества типа $G_{\delta \sigma}$.
			\item Счётные пересечения множеств типа $F_{\sigma}$~--- множества типа $F_{\sigma \delta}$.
		\end{enumerate}
	\end{st}


\end{document}

